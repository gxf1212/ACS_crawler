%!TEX root = document.tex

\section{安装与配置指南}
\label{sec:installation}

本章节详细介绍HA/OP自组装轨迹分析软件的安装过程、系统要求、环境配置以及初始设置步骤。软件采用模块化架构设计,支持多种操作系统,并提供完整的依赖管理和配置验证机制。

\subsection{系统要求与环境准备}
\label{subsec:system_requirements}

\paragraph{硬件要求}

\textbf{最低配置:}
\begin{itemize}
    \item CPU:Intel Core i5 或 AMD Ryzen 5 及以上,支持多核并行处理
    \item 内存:8 GB RAM(推荐16 GB以上用于大型轨迹分析)
    \item 存储:至少10 GB可用磁盘空间(轨迹数据和缓存文件需额外空间)
    \item 网络:稳定的互联网连接(用于软件包下载和更新)
\end{itemize}

\textbf{推荐配置:}
\begin{itemize}
    \item CPU:Intel Core i7/i9 或 AMD Ryzen 7/9,16核心及以上
    \item 内存:32 GB RAM 或更高
    \item 存储:SSD固态硬盘,100 GB以上可用空间
    \item GPU:CUDA兼容显卡(可选,用于加速某些计算)
\end{itemize}

\paragraph{操作系统支持}

软件支持以下操作系统环境:
\begin{itemize}
    \item \textbf{Linux:} Ubuntu 18.04+, CentOS 7+, RHEL 8+(推荐)
    \item \textbf{macOS:} macOS 10.14+ (Mojave及以上版本)
    \item \textbf{Windows:} Windows 10/11(通过WSL2或原生支持)
\end{itemize}

\paragraph{核心依赖软件}

\textbf{必需软件组件:}
\begin{itemize}
    \item \textbf{Python 3.9+:} 核心运行环境
    \item \textbf{GROMACS 2020+:} 分子动力学分析工具包
    \item \textbf{Git:} 版本控制和代码获取
    \item \textbf{XeLaTeX:} 文档编译(可选,仅文档生成需要)
\end{itemize}

\subsection{软件获取与安装}
\label{subsec:software_installation}

\paragraph{源码获取}

软件提供多种获取方式:

\textbf{方式一:Git克隆(推荐)}
\begin{lstlisting}[language=bash,style=blockstyle]
# 克隆主仓库
git clone https://github.com/username/HA_OP_delivery.git
cd HA_OP_delivery

# 检查版本信息
git tag --list
git checkout v1.0.0  # 使用稳定版本
\end{lstlisting}

\textbf{方式二:发布包下载}
\begin{lstlisting}[language=bash,style=blockstyle]
# 下载最新发布版本
wget https://github.com/username/HA_OP_delivery/archive/v1.0.0.tar.gz
tar -xzf v1.0.0.tar.gz
cd HA_OP_delivery-1.0.0
\end{lstlisting}

\paragraph{Python环境配置}

\textbf{虚拟环境创建:}
\begin{lstlisting}[language=bash,style=blockstyle]
# 创建专用虚拟环境
python -m venv ha_op_env

# 激活虚拟环境
# Linux/macOS:
source ha_op_env/bin/activate
# Windows (WSL):
source ha_op_env/Scripts/activate

# 验证Python版本
python --version  # 应显示3.9+
\end{lstlisting}

\textbf{依赖包安装:}
\begin{lstlisting}[language=bash,style=blockstyle]
# 升级pip到最新版本
pip install --upgrade pip

# 安装核心依赖
pip install -r requirements.txt

# 验证关键包安装
python -c "import mdanalysis; print(f'MDAnalysis: {mdanalysis.__version__}')"
python -c "import numpy; print(f'NumPy: {numpy.__version__}')"
python -c "import matplotlib; print(f'Matplotlib: {matplotlib.__version__}')"
\end{lstlisting}

\paragraph{GROMACS集成配置}

软件需要GROMACS工具包支持,配置方法如下:

\textbf{Linux系统(Ubuntu/Debian):}
\begin{lstlisting}[language=bash,style=blockstyle]
# 安装GROMACS
sudo apt update
sudo apt install gromacs

# 验证安装
gmx --version
which gmx  # 确认可执行文件路径
\end{lstlisting}

\textbf{通过源码编译安装:}
\begin{lstlisting}[language=bash,style=blockstyle]
# 下载GROMACS源码
wget http://ftp.gromacs.org/gromacs/gromacs-2023.tar.gz
tar -xzf gromacs-2023.tar.gz
cd gromacs-2023

# 编译安装
mkdir build && cd build
cmake .. -DCMAKE_INSTALL_PREFIX=/usr/local/gromacs
make -j$(nproc)
sudo make install

# 设置环境变量
echo 'source /usr/local/gromacs/bin/GMXRC' >> ~/.bashrc
source ~/.bashrc
\end{lstlisting}

\subsection{配置文件设置}
\label{subsec:configuration_setup}

\paragraph{基础配置文件}

软件提供多套预配置模板,用户可根据需求选择:

\begin{lstlisting}[language=bash,style=blockstyle]
# 查看可用配置模板
ls MD_assembly/coarse-grained-newOP-final/example_configs/

# 复制基础配置模板
cp MD_assembly/coarse-grained-newOP-final/example_configs/config_user.yaml \
   my_config.yaml
\end{lstlisting}

\textbf{配置文件基本结构:}
\begin{lstlisting}[language=yaml,style=blockstyle]
# 基本组定义
groups:
  "HA only":
    - 45nm-HA-30-e78.4-1
    - 45nm-HA-30-e78.4-2
    - 45nm-HA-30-e78.4-3

# 分析设置
analysis:
  types: ["rg", "ree", "clusters", "rdf", "contacts"]
  colors: ["#99D9D8", "#F1CACB", "#B8B8B8"]
  
  parameters:
    num_residues_per_molecule1: 36    # HA分子残基数
    num_residues_per_molecule2: 30    # OP分子残基数
    n_jobs: 4                         # 并行进程数
    use_parallel: true                # 启用并行处理

# 文件路径设置
files:
  topology: "solion.tpr"              # 拓扑文件名
  trajectory: "md_pbcmol.xtc"         # 轨迹文件名

# 目录设置
directories:
  base_path: "."                      # 数据基础路径
  cache_in_trajectory: true           # 缓存保存位置
  figure_output: "figures"            # 图形输出目录
\end{lstlisting}

\paragraph{路径配置}

根据系统环境调整路径设置:

\begin{lstlisting}[language=yaml,style=blockstyle]
# 示例:Linux环境配置
directories:
  base_path: "/home/user/MD_assembly/coarse-grained-newOP-final"
  figure_output: "/home/user/results/figures"
  
# 示例:Windows环境配置
directories:
  base_path: "C:\\Users\\username\\MD_assembly\\coarse-grained-newOP-final"
  figure_output: "C:\\Users\\username\\results\\figures"
\end{lstlisting}

\subsection{安装验证与测试}
\label{subsec:installation_verification}

\paragraph{基础功能测试}

\textbf{环境验证脚本:}
\begin{lstlisting}[language=bash,style=blockstyle]
# 创建验证脚本
cat > verify_installation.py << 'EOF'
#!/usr/bin/env python
"""安装验证脚本"""
import sys
import subprocess

def check_python_packages():
    """检查Python依赖包"""
    required_packages = [
        'numpy', 'pandas', 'matplotlib', 'mdanalysis', 
        'scipy', 'seaborn', 'joblib', 'tqdm', 'pyyaml'
    ]
    
    for package in required_packages:
        try:
            __import__(package)
            print(f"✓ {package} - 已安装")
        except ImportError:
            print(f"✗ {package} - 未安装")
            return False
    return True

def check_gromacs():
    """检查GROMACS安装"""
    try:
        result = subprocess.run(['gmx', '--version'], 
                              capture_output=True, text=True)
        if result.returncode == 0:
            print("✓ GROMACS - 已安装")
            return True
    except FileNotFoundError:
        pass
    
    print("✗ GROMACS - 未安装或未在PATH中")
    return False

def main():
    print("=== HA/OP分析软件安装验证 ===")
    print(f"Python版本: {sys.version}")
    
    py_ok = check_python_packages()
    gmx_ok = check_gromacs()
    
    if py_ok and gmx_ok:
        print("\n✓ 安装验证通过!")
        return 0
    else:
        print("\n✗ 安装验证失败,请检查缺失组件")
        return 1

if __name__ == "__main__":
    sys.exit(main())
EOF

# 运行验证
python verify_installation.py
\end{lstlisting}

\paragraph{示例数据测试}

\textbf{快速功能测试:}
\begin{lstlisting}[language=bash,style=blockstyle]
# 使用示例配置进行测试
python analyze_single.py --config example_configs/config_quick_test.yaml \
    --group "test" --analysis-types rg --n-jobs 2

# 检查输出结果
ls -la test_output/figures/
ls -la test_output/cache/
\end{lstlisting}

\paragraph{性能基准测试}

\begin{lstlisting}[language=bash,style=blockstyle]
# 创建性能测试脚本
cat > benchmark_test.py << 'EOF'
#!/usr/bin/env python
"""性能基准测试脚本"""
import time
import psutil
import numpy as np

def benchmark_parallel_performance():
    """测试并行处理性能"""
    print("CPU信息:")
    print(f"  核心数: {psutil.cpu_count(logical=False)}")
    print(f"  逻辑处理器: {psutil.cpu_count(logical=True)}")
    print(f"  内存总量: {psutil.virtual_memory().total // (1024**3)} GB")
    
    # 模拟计算任务
    calc_start = time.time()
    data = np.random.random((10000, 10000))
    result = np.linalg.eigvals(data[:100, :100])
    end_time = time.time()
    
    print(f"计算性能测试用时: {end_time - calc_start:.2f} 秒")

if __name__ == "__main__":
    benchmark_parallel_performance()
EOF

python benchmark_test.py
\end{lstlisting}

\subsection{常见安装问题与解决方案}
\label{subsec:troubleshooting}

\paragraph{依赖包冲突}

\textbf{问题:} Python包版本冲突或依赖解析失败
\begin{lstlisting}[language=bash,style=blockstyle]
# 清理环境重新安装
pip freeze > old_requirements.txt
pip uninstall -r old_requirements.txt -y
pip install -r requirements.txt

# 或创建全新环境
deactivate
rm -rf ha_op_env
python -m venv ha_op_env_new
source ha_op_env_new/bin/activate
pip install -r requirements.txt
\end{lstlisting}

\paragraph{GROMACS路径问题}

\textbf{问题:} 系统找不到GROMACS可执行文件
\begin{lstlisting}[language=bash,style=blockstyle]
# 查找GROMACS安装位置
find /usr -name "gmx" 2>/dev/null
find /opt -name "gmx" 2>/dev/null

# 添加到PATH环境变量
echo 'export PATH="/usr/local/gromacs/bin:$PATH"' >> ~/.bashrc
source ~/.bashrc

# 或创建软链接
sudo ln -s /usr/local/gromacs/bin/gmx /usr/local/bin/gmx
\end{lstlisting}

\paragraph{内存不足问题}

\textbf{问题:} 大型轨迹文件分析时内存溢出
\begin{lstlisting}[language=yaml,style=blockstyle]
# 在配置文件中限制资源使用
analysis:
  parameters:
    n_jobs: 2               # 降低并行进程数
    last_frames: 500        # 限制分析帧数
    use_parallel: false     # 禁用并行处理(如果必要)
\end{lstlisting}

\paragraph{文件权限问题}

\textbf{问题:} 无法创建缓存文件或图形输出
\begin{lstlisting}[language=bash,style=blockstyle]
# 检查目录权限
ls -la MD_assembly/
chmod -R 755 MD_assembly/

# 创建必要目录
mkdir -p results/figures results/cache
chmod 755 results/figures results/cache
\end{lstlisting}

\subsection{更新与维护}
\label{subsec:maintenance}

\paragraph{软件更新}

\begin{lstlisting}[language=bash,style=blockstyle]
# 检查当前版本
git describe --tags

# 拉取最新更新
git fetch --tags
git pull origin main

# 更新依赖包
pip install -r requirements.txt --upgrade

# 重新验证安装
python verify_installation.py
\end{lstlisting}

\paragraph{配置文件迁移}

\begin{lstlisting}[language=bash,style=blockstyle]
# 备份现有配置
cp my_config.yaml my_config.yaml.backup

# 检查配置文件兼容性
python -c "import yaml; yaml.safe_load(open('my_config.yaml'))"
\end{lstlisting}

通过完成以上安装配置步骤,用户可以获得完全功能的HA/OP自组装轨迹分析环境,为后续的科学研究和数据分析奠定坚实基础。
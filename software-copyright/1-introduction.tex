%!TEX root = document.tex

\section{引言}
\label{sec:introduction_main}

\subsection{编写目的}
\label{ssec:intro_purpose}
本软件著作权说明文档旨在明确描述"\Name"(版本号 \Version)的研发背景、核心功能模块、技术特点及具体使用方法。本文档的目的是为用户、开发人员及相关评估机构提供全面、准确的技术参考,确保软件的正确理解、高效使用与合规评估,并作为申请软件著作权的必要支撑材料。同时,本文档也为后续的软件维护、功能迭代与技术交流奠定基础。

\subsection{软件简介与主要功能}
\label{ssec:intro_summary_features}
"\Name"是一个专业的美国化学会(ACS)出版物论文网络爬虫系统,具有现代化的Web仪表板和数据分析功能。该软件基于Python生态系统开发,集成了Selenium自动化爬虫技术、FastAPI Web框架和SQLite数据库,为科研人员和学术机构提供高效、可靠的学术论文数据采集解决方案。

软件采用模块化架构设计,支持后台异步任务处理、实时进度跟踪和数据可视化,通过直观的Web界面为用户提供便捷的论文检索、筛选和管理功能。

\subsubsection{核心功能模块}
\label{sssec:core_modules}

\textbf{论文爬取系统(Paper Scraper):}
\begin{itemize}
    \item 支持43种预配置的ACS学术期刊,覆盖化学、材料科学、生物学等领域
    \item 基于Selenium的智能浏览器自动化,有效绕过反爬虫机制
    \item 完整的元数据提取,包括标题、DOI、作者、摘要、关键词、引用信息等
    \item 智能错误处理和重试机制,确保爬取过程的稳定性
\end{itemize}

\textbf{Web仪表板系统(Dashboard):}
\begin{itemize}
    \item 现代化的响应式Web界面,基于Bootstrap 5框架
    \item 实时数据可视化,集成Chart.js图表库
    \item 高级搜索和筛选功能,支持按标题、作者、期刊、年份等多维度过滤
    \item 详细的论文信息展示页面,包含完整的学术元数据
\end{itemize}

\textbf{后台任务管理系统:}
\begin{itemize}
    \item 异步任务处理架构,支持多个爬取任务并发执行
    \item 实时进度跟踪和任务状态监控
    \item 任务取消和日志查看功能
    \item 智能资源管理,避免系统过载
\end{itemize}

\textbf{数据存储与检索系统:}
\begin{itemize}
    \item 基于SQLite的关系型数据库,支持复杂查询和数据分析
    \item 完整的数据模型设计,包括论文、作者、关键词、爬取任务等实体
    \item 全文搜索功能,支持快速检索论文内容
    \item 数据导出功能,支持CSV、Excel等多种格式
\end{itemize}

\textbf{RESTful API系统:}
\begin{itemize}
    \item 完整的REST API接口,支持第三方系统集成
    \item 自动生成的API文档,位于\lstinline|/docs|路径
    \item 标准化的HTTP响应格式,支持JSON数据交换
    \item 完善的错误处理和状态码管理
\end{itemize}

\subsection{技术特色与优势}
\label{ssec:technical_features}

\textbf{现代化技术栈:} 采用Python 3.9+、FastAPI、Selenium、Bootstrap 5等现代Web技术,确保系统的高性能和可维护性。

\textbf{反爬虫绕过:} 集成Selenium WebDriver和WebDriver Manager,自动管理浏览器驱动,有效应对网站的自动化检测机制。

\textbf{异步处理架构:} 基于FastAPI的异步框架设计,支持高并发请求处理,提升用户体验。

\textbf{模块化设计:} 采用松耦合的模块化架构,便于功能扩展和定制化开发,各组件职责明确,接口清晰。

\textbf{用户友好界面:} 提供直观的Web界面,无需编程知识即可完成复杂的论文爬取和管理任务。

\textbf{数据完整性:} 确保爬取数据的完整性和准确性,提供数据验证和错误恢复机制。

\subsection{适用对象}
\label{ssec:intro_target_audience}
本软件主要适用于以下用户群体及研究领域:

\begin{itemize}
    \item \textbf{科研人员:} 从事化学、材料科学、生物学、药学等领域研究的学者和研究生
    \item \textbf{学术机构:} 大学、研究所、实验室等需要进行文献调研和数据收集的机构
    \item \textbf{图书情报人员:} 负责学术资源管理和文献服务的图书管理员和信息专家
    \item \textbf{数据分析师:} 需要大量学术数据进行计量分析和趋势研究的专业人员
    \item \textbf{教育工作者:} 进行学术写作指导和科研方法培训的教师和导师
\end{itemize}

用户通常需要具备基础的计算机操作技能,了解学术文献检索的基本概念,但对于编程技能没有强制要求。

\subsection{软件部署与技术支持}
\label{ssec:intro_deployment_support}
本软件基于跨平台的Python生态系统开发,支持Linux、macOS和Windows等主流操作系统。软件提供简单的部署方式和完善的文档支持。

\textbf{技术要求:}
\begin{itemize}
    \item Python 3.9及以上版本
    \item Chrome浏览器及ChromeDriver(自动下载)
    \item 现代Web浏览器(用于访问Web界面)
    \item 推荐4GB以上内存用于正常运行
\end{itemize}

\textbf{获取方式:}
软件源代码托管于GitHub平台,用户可通过克隆仓库或下载发布包的方式获取。

\textbf{技术支持:}
提供完整的文档系统,包括安装指南、使用说明、API文档和常见问题解答。用户可通过GitHub Issues页面提交技术问题和功能建议。

\subsection{法律合规与使用限制}
\label{ssec:intro_legal_compliance}
本软件专为学生、学者、教育工作者和研究人员设计,用于学术研究、教育和数据收集目的。

\textbf{允许用途:}
\begin{itemize}
    \item 学术研究和教育目的
    \item 个人学习和技能提升
    \item 非商业性的数据收集和分析
\end{itemize}

\textbf{使用限制:}
\begin{itemize}
    \item 禁止商业用途,除非获得明确许可
    \item 必须遵守ACS出版物的服务条款和使用政策
    \item 不得用于违反版权法或相关法律法规的活动
    \item 用户需自行承担使用软件的法律责任
\end{itemize}

\textbf{免责声明:}
本软件仅供教育和研究目的使用。开发者不对软件使用可能导致的任何法律问题承担责任。用户在使用软件前应仔细阅读并遵守相关法律法规和网站服务条款。

\subsection{文档结构说明}
\label{ssec:document_structure}
本文档按照软件功能模块和使用流程组织,主要包括:
\begin{itemize}
    \item \textbf{第2节}:软件安装配置与环境准备
    \item \textbf{第3节}:核心功能使用方法与操作指南
    \item \textbf{第4节}:系统架构设计与技术实现
\end{itemize}

每个章节都包含详细的使用说明、技术架构介绍和实际应用示例,帮助用户全面了解软件的功能特性和使用方法。